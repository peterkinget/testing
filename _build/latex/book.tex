%% Generated by Sphinx.
\def\sphinxdocclass{jupyterBook}
\documentclass[letterpaper,10pt,english]{jupyterBook}
\ifdefined\pdfpxdimen
   \let\sphinxpxdimen\pdfpxdimen\else\newdimen\sphinxpxdimen
\fi \sphinxpxdimen=.75bp\relax
\ifdefined\pdfimageresolution
    \pdfimageresolution= \numexpr \dimexpr1in\relax/\sphinxpxdimen\relax
\fi
%% let collapsible pdf bookmarks panel have high depth per default
\PassOptionsToPackage{bookmarksdepth=5}{hyperref}
%% turn off hyperref patch of \index as sphinx.xdy xindy module takes care of
%% suitable \hyperpage mark-up, working around hyperref-xindy incompatibility
\PassOptionsToPackage{hyperindex=false}{hyperref}
%% memoir class requires extra handling
\makeatletter\@ifclassloaded{memoir}
{\ifdefined\memhyperindexfalse\memhyperindexfalse\fi}{}\makeatother

\PassOptionsToPackage{booktabs}{sphinx}
\PassOptionsToPackage{colorrows}{sphinx}

\PassOptionsToPackage{warn}{textcomp}

\catcode`^^^^00a0\active\protected\def^^^^00a0{\leavevmode\nobreak\ }
\usepackage{cmap}
\usepackage{fontspec}
\defaultfontfeatures[\rmfamily,\sffamily,\ttfamily]{}
\usepackage{amsmath,amssymb,amstext}
\usepackage{polyglossia}
\setmainlanguage{english}



\setmainfont{FreeSerif}[
  Extension      = .otf,
  UprightFont    = *,
  ItalicFont     = *Italic,
  BoldFont       = *Bold,
  BoldItalicFont = *BoldItalic
]
\setsansfont{FreeSans}[
  Extension      = .otf,
  UprightFont    = *,
  ItalicFont     = *Oblique,
  BoldFont       = *Bold,
  BoldItalicFont = *BoldOblique,
]
\setmonofont{FreeMono}[
  Extension      = .otf,
  UprightFont    = *,
  ItalicFont     = *Oblique,
  BoldFont       = *Bold,
  BoldItalicFont = *BoldOblique,
]



\usepackage[Bjarne]{fncychap}
\usepackage[,numfigreset=1,mathnumfig]{sphinx}

\fvset{fontsize=\small}
\usepackage{geometry}


% Include hyperref last.
\usepackage{hyperref}
% Fix anchor placement for figures with captions.
\usepackage{hypcap}% it must be loaded after hyperref.
% Set up styles of URL: it should be placed after hyperref.
\urlstyle{same}


\usepackage{sphinxmessages}



        % Start of preamble defined in sphinx-jupyterbook-latex %
         \usepackage[Latin,Greek]{ucharclasses}
        \usepackage{unicode-math}
        % fixing title of the toc
        \addto\captionsenglish{\renewcommand{\contentsname}{Contents}}
        \hypersetup{
            pdfencoding=auto,
            psdextra
        }
        % End of preamble defined in sphinx-jupyterbook-latex %
        

\title{EE4312 Analog Electronic Circuits Notes}
\date{Sep 07, 2024}
\release{}
\author{Peter Kinget}
\newcommand{\sphinxlogo}{\vbox{}}
\renewcommand{\releasename}{}
\makeindex
\begin{document}

\pagestyle{empty}
\sphinxmaketitle
\pagestyle{plain}
\sphinxtableofcontents
\pagestyle{normal}
\phantomsection\label{\detokenize{intro::doc}}


\sphinxstepscope


\chapter{Pre\sphinxhyphen{}Requisites Test for Analog Electronic Circuits}
\label{\detokenize{TestYourKnowledge/AEC_prereq_test/AEC_entry_test:pre-requisites-test-for-analog-electronic-circuits}}\label{\detokenize{TestYourKnowledge/AEC_prereq_test/AEC_entry_test::doc}}

\section{Assumed Preparation}
\label{\detokenize{TestYourKnowledge/AEC_prereq_test/AEC_entry_test:assumed-preparation}}\begin{itemize}
\item {} 
\sphinxAtStartPar
EE3201 Circuit Analysis

\item {} 
\sphinxAtStartPar
EE3106 Solid\sphinxhyphen{}State Devices and Materials

\item {} 
\sphinxAtStartPar
EE3331 Electronic Circuits I

\item {} 
\sphinxAtStartPar
EE3081 Signals and Systems

\end{itemize}


\section{Circuit Analysis}
\label{\detokenize{TestYourKnowledge/AEC_prereq_test/AEC_entry_test:circuit-analysis}}\begin{enumerate}
\sphinxsetlistlabels{\arabic}{enumi}{enumii}{}{.}%
\item {} 
\sphinxAtStartPar
Analyze the following circuits:

\end{enumerate}

\sphinxAtStartPar
\sphinxincludegraphics{{RC_Bode1}.png}
\begin{itemize}
\item {} 
\sphinxAtStartPar
Derive the expression for the transfer function \(H(s) = V_{out}(s)/V_{in}(s)\).
\begin{itemize}
\item {} 
\sphinxAtStartPar
What are the poles and zeros of \(H(s)\)?

\item {} 
\sphinxAtStartPar
Plot the Bode diagram for \(H(s)\) assuming \(R_1 = 10K\Omega, C_1 = 160pF, R_2 = 1K\Omega\).

\end{itemize}

\item {} 
\sphinxAtStartPar
Assume \(V_{in}(t) = \sin(2\pi f t)\) with f = 100kHz; plot \(V_{in}(t)\) and \(V_{out}(t)\) in steady state.

\item {} 
\sphinxAtStartPar
What type of filter is the network on the left?

\item {} 
\sphinxAtStartPar
The network on the right is called a \sphinxstyleemphasis{lag} network; can you explain why?

\end{itemize}
\begin{enumerate}
\sphinxsetlistlabels{\arabic}{enumi}{enumii}{}{.}%
\setcounter{enumi}{1}
\item {} 
\sphinxAtStartPar
Repeat the same analysis for the following circuits:

\end{enumerate}

\sphinxAtStartPar
\sphinxincludegraphics{{RC_Bode2}.png}
\begin{itemize}
\item {} 
\sphinxAtStartPar
Now assume \(R_1 = 10K\Omega, C_1 = 160pF, R_2 = 100K\Omega\).

\item {} 
\sphinxAtStartPar
The network on the right is called a \sphinxstyleemphasis{lead} network; can you explain why?

\end{itemize}
\begin{enumerate}
\sphinxsetlistlabels{\arabic}{enumi}{enumii}{}{.}%
\setcounter{enumi}{2}
\item {} 
\sphinxAtStartPar
Determine the impedance \(Z(s) = V_{test}(s)/I_{test}(s)\) for the following networks:

\end{enumerate}

\sphinxAtStartPar
\sphinxincludegraphics{{impedances}.png}‘
\begin{itemize}
\item {} 
\sphinxAtStartPar
Plot the magnitude \(|Z(j\omega)|\) vs the angular frequency \(\omega\) on a \sphinxstyleemphasis{log\sphinxhyphen{}log} plot and the phase \(\angle Z(j\omega)\) vs the angular frequency \(\omega\) on \sphinxstyleemphasis{semilogx} plot over the relevant frequency range.
\begin{itemize}
\item {} 
\sphinxAtStartPar
Assume \(R_1\) = 10\(\Omega\), \(R_2\) = 1\(K\Omega\), and \(L_1\) = 160\(\mu H\) for the left network

\item {} 
\sphinxAtStartPar
Assume \(R_1\) = 10\(\Omega\), \(R_2\) = 1\(K\Omega\), and \(C_1\) = 1.6\(nF\) for the right network

\end{itemize}

\end{itemize}
\begin{enumerate}
\sphinxsetlistlabels{\arabic}{enumi}{enumii}{}{.}%
\setcounter{enumi}{3}
\item {} 
\sphinxAtStartPar
Use a spice\sphinxhyphen{}type simulator to verify your results. Carefully consider which simulation analyses to use for the various parts of the questions.

\end{enumerate}


\section{Systems Analysis}
\label{\detokenize{TestYourKnowledge/AEC_prereq_test/AEC_entry_test:systems-analysis}}\begin{enumerate}
\sphinxsetlistlabels{\arabic}{enumi}{enumii}{}{.}%
\item {} 
\sphinxAtStartPar
For the following transfer functions, are the poles and zeros in the left or right half of the complex plane? Draw their impulse response and step response:
\begin{itemize}
\item {} 
\sphinxAtStartPar
\(H(s) = \frac{A}{1 + s/\omega_p}\)
\begin{itemize}
\item {} 
\sphinxAtStartPar
with \(\omega_p = 1/(2\pi 1\mu s)\)

\end{itemize}

\item {} 
\sphinxAtStartPar
\(H(s) = \frac{A}{1 + s/\omega_p}\)
\begin{itemize}
\item {} 
\sphinxAtStartPar
with \(\omega_p = -1/(2\pi 1\mu s)\)

\end{itemize}

\item {} 
\sphinxAtStartPar
\(H(s) = \frac{A}{(1 + s/s_{1})(1 + s/s_{2})}\)
\begin{itemize}
\item {} 
\sphinxAtStartPar
with \(s_1 = \alpha + j \omega_{p}\), \(s_2 = \alpha - j \omega_{p}\), \(\alpha = 1/(10 \mu s)\), \(\omega_p = 1/(2\pi 1\mu s)\)

\end{itemize}

\item {} 
\sphinxAtStartPar
\(H(s) = \frac{A}{(1 + s/s_{1})(1 + s/s_{2})}\)
\begin{itemize}
\item {} 
\sphinxAtStartPar
with \(s_1 = \alpha + j \omega_{p}\), \(s_2 = \alpha - j \omega_{p}\), \(\alpha = -1/(10 \mu s)\), \(\omega_p = 1/(2\pi 1\mu s)\)

\end{itemize}

\end{itemize}

\item {} 
\sphinxAtStartPar
(\sphinxstyleemphasis{more advanced}) Derive the expression for the impulse and step response for the circuits above and plot them.

\item {} 
\sphinxAtStartPar
(\sphinxstyleemphasis{more advanced}) Use a toolbox in a mathematical tool like Matlab or NumPy (with the Control Systems Library) to check your results

\end{enumerate}


\section{MOS Devices and Basic Electronics Circuits}
\label{\detokenize{TestYourKnowledge/AEC_prereq_test/AEC_entry_test:mos-devices-and-basic-electronics-circuits}}
\sphinxAtStartPar
You can assume a 5V CMOS technology with transistor threshold voltages \(V_{Tn} = |V_{Tp}| = \) 0.7V; the transistors have ideal square\sphinxhyphen{}law behavior.
\begin{enumerate}
\sphinxsetlistlabels{\arabic}{enumi}{enumii}{}{.}%
\item {} 
\sphinxAtStartPar
Given the following two\sphinxhyphen{}transistor circuit:

\end{enumerate}

\sphinxAtStartPar
\sphinxincludegraphics{{inverter2}.png}
\begin{itemize}
\item {} 
\sphinxAtStartPar
Plot the \(V_{OUT}\)\sphinxhyphen{}\(V_{IN}\) characteristic for \(V_{IN}\) going from 0 to 5V.

\item {} 
\sphinxAtStartPar
What the region of operation of \(M_1\) and \(M_2\), when \(V_{IN}\) is 0, 2.5 and 5V; explain your reasoning.

\end{itemize}
\begin{enumerate}
\sphinxsetlistlabels{\arabic}{enumi}{enumii}{}{.}%
\setcounter{enumi}{1}
\item {} 
\sphinxAtStartPar
Given the following one\sphinxhyphen{}transistor amplifier%
\begin{footnote}[1]\sphinxAtStartFootnote
In practical circuits we would never bias this amplifier with a fixed voltage; this is done here to keep the schematic simpler.
%
\end{footnote}  \(V_B = \) 0.9V, and the nMOS transistor characteristics shown:

\end{enumerate}

\sphinxAtStartPar
\sphinxincludegraphics{{CS_amp}.png}
\begin{itemize}
\item {} 
\sphinxAtStartPar
Assuming \(V_{IN} = 0\),
\begin{itemize}
\item {} 
\sphinxAtStartPar
determine the bias current through M1

\item {} 
\sphinxAtStartPar
determine the DC bias of \(V_{OUT}\)

\item {} 
\sphinxAtStartPar
what is the gate\sphinxhyphen{}overdrive voltage of M1

\item {} 
\sphinxAtStartPar
what is the transconductance \(g_m\) of M1; if you cannot find the \(g_m\), use 2mS for the remainder of this question.

\end{itemize}

\item {} 
\sphinxAtStartPar
Assuming \(V_{IN} = 10mV\),
\begin{itemize}
\item {} 
\sphinxAtStartPar
what are the current through M1 and \(V_{OUT}\) now?

\item {} 
\sphinxAtStartPar
based on this calculation, what is the small\sphinxhyphen{}signal gain of the amplifier?

\end{itemize}

\end{itemize}


\bigskip\hrule\bigskip


\sphinxstepscope


\chapter{SOLUTION: Pre\sphinxhyphen{}Requisites Test for Analog Electronic Circuits}
\label{\detokenize{TestYourKnowledge/AEC_prereq_solution/AEC_entry_test_solution:solution-pre-requisites-test-for-analog-electronic-circuits}}\label{\detokenize{TestYourKnowledge/AEC_prereq_solution/AEC_entry_test_solution::doc}}

\section{Circuit Analysis}
\label{\detokenize{TestYourKnowledge/AEC_prereq_solution/AEC_entry_test_solution:circuit-analysis}}\begin{enumerate}
\sphinxsetlistlabels{\arabic}{enumi}{enumii}{}{.}%
\item {} 
\sphinxAtStartPar
Analysis of RC Networks

\end{enumerate}
\begin{itemize}
\item {} 
\sphinxAtStartPar
Bode Plots of transfer function \(H(j 2\pi f)\)
\begin{itemize}
\item {} 
\sphinxAtStartPar
Small Signal AC analysis \sphinxcode{\sphinxupquote{.ac}}

\end{itemize}

\end{itemize}

\sphinxAtStartPar
\sphinxincludegraphics{{schematic_H_ac}.png}
\sphinxincludegraphics{{results_H_ac}.png}
\begin{itemize}
\item {} 
\sphinxAtStartPar
Response to a Sine Wave
\begin{itemize}
\item {} 
\sphinxAtStartPar
Transient simulation \sphinxcode{\sphinxupquote{.tran}} with a sinusoidal input signal

\end{itemize}

\end{itemize}

\sphinxAtStartPar
\sphinxincludegraphics{{schematic_H_sine}.png}
\sphinxincludegraphics{{results_H_sine_transient}.png}
\sphinxincludegraphics{{results_H_sine_steady_state}.png}
\begin{itemize}
\item {} 
\sphinxAtStartPar
Extra: Impulse Response
\begin{itemize}
\item {} 
\sphinxAtStartPar
Transient simulation \sphinxcode{\sphinxupquote{.tran}} with a step input created with a PWL source

\end{itemize}

\end{itemize}

\sphinxAtStartPar
\sphinxincludegraphics{{schematic_H_impulse}.png}
\sphinxincludegraphics{{results_H_impulse}.png}
\begin{enumerate}
\sphinxsetlistlabels{\arabic}{enumi}{enumii}{}{.}%
\setcounter{enumi}{2}
\item {} 
\sphinxAtStartPar
Analysis of Impedance

\end{enumerate}
\begin{itemize}
\item {} 
\sphinxAtStartPar
Bode Plots of transfer function \(Z(j 2\pi f)\)
\begin{itemize}
\item {} 
\sphinxAtStartPar
Small Signal AC analysis \sphinxcode{\sphinxupquote{.ac}}

\end{itemize}

\end{itemize}

\sphinxAtStartPar
\sphinxincludegraphics{{schematic_Z}.png}
\sphinxincludegraphics{{results_Z}.png}







\renewcommand{\indexname}{Index}
\printindex
\end{document}